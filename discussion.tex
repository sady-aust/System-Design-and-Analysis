For developing ‘Biponee’ we follow different phases.At first we analyze the project feasibility. Here we analyze three kinds of project feasibility. They are operational feasibility, technical feasibility and economical feasibility. \\
After that we analyze cost benefit and present value and future value of the project finally we make a project scheduling chart at this step.\\
Then we go to information gathering phase. In this phase we collect information from our client “Reverie Bangladesh” and gain the complete idea about how they manage their current system. \\
In the next step we design the data flow diagram and Use case diagram. From data flow diagram we identified what kind of data will be the input and output of our project, how data will advance in this system and where the data will be stored. From use case diagram we identify the use case and the actors of our project.\\
Then we design the ER diagram and class diagram. In ER diagram we identify the entities and relationships of our project. The ER diagram provide the visual starting point of our project database. The class diagram defines the method and variables of an object. This class diagram is useful in all forms of object oriented programming.\\\\
The technologies we used in frontend development are:
\begin{itemize}
    \item HTML
    \item CSS
    \item Javascript
    \item jQuery
    \item AJAX
\end{itemize}
\\\\
The technologies we used in backend are:
\begin{itemize}
    \item C#
    \item SQL
    
\end{itemize}
\\\\

The IDE we used in this project development:
\begin{itemize}
    \item Microsoft Visual Studio 2018
    \item Microsoft SQL Server Management Studio 17
    
\end{itemize}
\\\\

The architecture we used:
\begin{itemize}
    \item MVC Architecture
    \item Layer Architecture
    
\end{itemize}
\\\\
The framework we used:
\begin{itemize}
    \item Asp.net MVC Framework
\end{itemize}
\\
In the future we will add more section in our project. Try to implement Artificial intelligence(AI), and Machine learning to make our project more efficient. 




