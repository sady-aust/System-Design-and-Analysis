
  
\section{Selection the objectives of Information Gathering & Interview}

``Reverie Bangladesh" is a clothing (brand) which are currently selling various kinds of t-shirts through their Facebook page. After getting better response from customers and clients, they want to grow up their business.\\\\
So they offered us to develop a web based system named ``Biponee" where they can sell variety of products such as men's clothing, women's clothing, watch and jewelry, electronics etc.\\\\
So, to know about how their existing system works, how they maintain it, we called an interview on 24 May, 2018. This interview helps us to gather information from them which is helpful for us to improve features to this project.

\section{Questionnaires and Interview pattern}
To know the client requirements, we need to gather information from them. There are various way to gather information. Some of them are describing below:\\

\textbf{Interview:} An interview is a conversation where questions are asked and answers are given. Interviews usually take place face to face and in person. We can modify it. In an interview we gather information by asking questions to the interviewee. There are two types of questions, which are:\\
 
  \underline{1. Open-ended Question:} Open-ended questions are ones that require more than one word answers. “Open” actually describes the interviewee’s options for responding. They are open. The response can be two words or two paragraphs. \\\\
\underline{2. Closed Question:} Closed  questions can be answered in only one word or with a short, specific piece of information. A closed question limits the response available to the interviewee. \\
 
\textbf {Questionnaire:} A questionnaire is a research instrument consisting of a series of questions (or other types of prompts) for the purpose of gathering information from respondents. Questionnaires are also sharply limited by the fact that respondents must be able to read the questions and respond to them. In this process we can not modify the question and it is not necessary to face to face interaction.\\

For information gathering we use interview method, because for our project we need to interact with the client face to face for analyze their requirements. We used Pyramid structure for our system.


 \section{Selection of interview Personnel}
 We have selected Mr. Rahee Zaman, founder of ``Reverie Bangladesh" for the interview. For our project it is necessary to know how they currently running their system, what are the difficulties they are facing to running this system, what are the requirements and features they want in ``Biponee". As a founder of "Reverie Bangladesh", Mr. Rahee Zaman knows everything about their existing system. So the whole interview was taken from him.
 
  \section{Summary of the total information and list of activities}
  After interviewing Mr. Rahee Zaman, we identify the drawbacks of their current system. We also know about some new features we need to integrate in ``Biponee". Besides they shared with us about the difficulties they are facing right now and about the future plan of Reverie Bangladesh.  \\
  
  The list of activities   are:
  \begin{itemize}
  \item Knowing about order management
  \item Knowing about delivery system
  \item Current payment system
  \item Current inventory management
  \item  Reasons behind shifting to BIPONEE
   \item Features 
   \item Future plan
\end{itemize}

\section{Conclusion}
After the information gathering process, we have a complete idea  about how they manage their sections, what they want in ``Biponee", how they take/serve the orders from/to the customer, communicate with the customer etc. Hopefully by our project ``Biponee" they will manage their business smartly, will achieve customer satisfaction, and grow up their business rapidly.
  
  
  
 
 




